% ----------------------------------------------------------------------------
% Vorlage Abschlussarbeit Informatik THM (minimal)
%
% Copyright (c) 2016 - 2020 by Burkhardt Renz. All rights reserved.
% Die Vorlage für eine Abschlussarbeit in der Informatik am Fachbereich
% MNI der THM ist lizenziert unter einer Creative Commons
% Namensnennung-Nicht kommerziell 4.0 International Lizenz.
%
% $Id: vorlage.tex 979 2020-08-24 07:03:19Z br $
% ----------------------------------------------------------------------------

\documentclass[%
	BCOR=8.25mm,         % Bindekorrektur
	DIV=12,              % Satzspiegel
	parskip=half,				 % Abstand zwischen Absätzen
	bibliography=totoc,	 % Literaturverzeichnis im Inhaltsverzeichnis
	headsepline=on,      % Trennlinie Kolumnentitel
	]{scrbook}

%% Präambel
\usepackage[english, ngerman]{babel} % deutsche typogr. Regeln + Trenntabelle
\usepackage[T1]{fontenc}             % interner TeX-Font-Codierung
\usepackage{helvet}                 % Font Latin Modern
\usepackage[utf8]{inputenc}          % Font-Codierung der Eingabedatei
\usepackage[babel]{csquotes}         % Anführungszeichen
\usepackage{graphicx}                % Graphiken
\usepackage{booktabs}                % Tabellen schöner
\usepackage{listingsutf8}            % Listings mit Einstellungen
\lstset{basicstyle=\small\ttfamily,
	tabsize=2,
	basewidth={0.5em,0.45em},
	extendedchars=true}
\lstset{literate=%                   % Umlaute in Listings
  {Ö}{{\"O}}1
  {Ä}{{\"A}}1
  {Ü}{{\"U}}1
  {ß}{{\ss}}2
  {ü}{{\"u}}1
  {ä}{{\"a}}1
  {ö}{{\"o}}1
	{~}{{\textasciitilde}}1} 
\usepackage{amsmath}	               % Mathematik
\usepackage[pdftex]{hyperref}       
\hypersetup{
	bookmarksopen=true,
	bookmarksopenlevel=3,
	colorlinks,
	citecolor=blue,
	linkcolor=blue,
}
\usepackage{scrhack}				% unterdrückt Fehlermeldung von listings
\usepackage[
	sorting = none,
	style = authoryear,
	backend = biber]{biblatex}   % Biblatex für das Literaturverzeichnis
\addbibresource{Hadoop.bib}			% Import der Bibliothek von Zotero

%% Nummerierungstiefen
\setcounter{tocdepth}{3}             % 3 Stufen im Inhaltsverzeichnis
\setcounter{secnumdepth}{3} 		     % 3 Stufen in Abschnittnummerierung

% ----------------------------------------------------------------------------
\begin{document}

\frontmatter

%% Titelseite
\begin{titlepage}
	\begin{center}
	\includegraphics[width=0.9\textwidth]{img/mni-logo}
	
	\vspace{5cm}	

	\vspace{1cm}	

	\huge\textbf{\sffamily Apache Hadoop}

	\normalsize
	\vspace{1cm}	

	Hauptseminar "Cloud-Plattformen und Big Data" \\
	Dozent Steffen Rupp

	von \\[1cm]	

	\textbf{René Gentzen}\\ [.5cm] 
	rene.gentzen@mni.thm.de\\ [.5cm] 
	im WS22/23
	\end{center}
	\vfill
\end{titlepage}

%% Verzeichnissse
\tableofcontents

\listoffigures
\listoftables
\lstlistoflistings

\mainmatter 
\pagestyle{headings}
\chapter{Hadoop Grundlagen}

``Die Apache Hadoop Softwarebibliothek ist ein Framework, das die über Computercluster verteilte Verarbeitung großer Datensätze mit einfachen Programmiermodellen ermöglicht. Es ist so konzipiert, dass es von einzelnen Servern bis hin zu Tausenden von Rechnern skaliert werden kann, von denen jeder lokale Rechenleistung und Speicherplatz bietet. Anstatt sich auf Hardware zu verlassen, um eine hohe Verfügbarkeit zu gewährleisten, ist die Bibliothek selbst so konzipiert, dass sie Ausfälle auf der Anwendungsebene erkennt und bewältigt, so dass ein hochverfügbarer Dienst auf einem Cluster von Computern bereitgestellt wird, von denen jeder für sich für Ausfälle anfällig sein kann.''\cite{noauthor_apache_nodate}\\
So beschreibt (übersetzt aus dem Englischen) die Apache Software Foundation ihr Top Level Projekt Apache™ Hadoop®.  Diese Arbeit wird einen pragmatischen Überblick über Hadoop und die Komponenten im Hadoop Ecosystem geben. Dabei soll der Fokus nicht auf technischen Details, sondern auf Anwendungsorientiertheit leigen. Es sollen konkrete Anwendungsfälle zu den einzelnen Komponenten besprochen werden und nicht abstrakte Architekturkonzepte.

\section{Hadoop Historie}
\subsection{Anforderungen von Big Data}
``Der Begriff „Big Data“ bezieht sich auf Datenbestände, die so groß, schnelllebig oder komplex sind, dass sie sich mit herkömmlichen Methoden nicht oder nur schwer verarbeiten lassen.''\cite{noauthor_big_nodate} \\ 
Schon Anfang der Neunziger war es nicht mehr praktikabel, Webseiten händisch, zum Beispiel in ''Web Directories'', zu katalogisieren. Man wollte Nutzern trotzdem die Möglichkeit geben, Informationen durch das Durchsuchen zentraler Anlaufstellen ausfindig zu machen. Automatisierte Tools, die sogenannten ''Web Crawler'' wurden erfunden, um diese Arbeit zu übernehmen.\cite{griffiths_search_2007} \\ 
Das Internet erlebte in den letzten Jahren des 20. Jahrhunderts ein explosionsartiges Wachstum an Nutzern und Webseiten, und damit auch an Informationen, die katalogisiert werden mussten.\cite{zakon_hobbes_2018} 
Um eine immer größer werdende Menge an Informationen verarbeiten zu können, gibt es zwei Ansätze der Skalierung: Vertikale und horizontale Skalierung. Diese sollen in den folgenden Abschnitten erläutert werden.

\subsection{Vertikale Skalierung}
Bei der vertikalen Skalierung (''scaling up'') werden \emph{einem} System mehr Ressourcen wie zum Beispiel größerer Speicher, oder eine schnellere CPU hinzugefügt. Dadurch bekommt man einen Performance-Gewinn: Man kann mehr Daten speichern, oder Berechnungen werden schneller fertig gestellt.
Ein großer Vorteil der vertikalen Skalierung ist, dass Anwendungsprogramme in der Regel nicht angepasst werden müssen, um vom diesem Performance-Wachstum zu profitieren. Wenn man eine 5TB große Festplatte gegen eine 10TB Festplatte austauscht, dann hat man den Speicherplatz eines Servers vertikal skaliert. Die darauf laufenden Programme müssen nicht angepasst werden, sondern man kann einfach doppelt so viele Daten speichern.\cite{beaumont_how_2014}\\
Vertikale Skalierung hat drei große Nachteile: Erstens kann man nicht unbegrenzt vertikal skalieren. Ein Server kann physisch nur eine begrenzte Anzahl an Hardware aufnehmen. Zweitens wächst die Performance eines Systems bei vertikaler Skalierung höchstens linear\cite{gustafson_amdahls_2011}, die Kosten allerdings nicht. Heutzutage kann man gerade bei Cloud-Anbietern sehr leistungsfähige Systeme bei linearem Preisanstieg mieten.\cite{noauthor_pricing_nodate} Sucht man aber noch mehr Performance in \emph{einem} System, dann steigen die Kosten exponentiell\cite{athow_at_2020}. Drittens skalieren nicht alle Faktoren in einem System gleich gut vertikal. Die Speicherkapazität von SSDs ist zum Beispiel seit 1978 von 45MB auf 100TB gestiegen (Faktor $2222,22 \cdot 10^{3}$), während sich die Datenrate nur von 1.5MB/s auf 500/460MB/s (Sequential Read/Write) erhöht hat (Faktor $0,333 \cdot 10^{3}$).\cite{noauthor_who_nodate}\cite{athow_at_2020}  


\subsection{Horizontale Skalierung}
Bei der horizontalen Skalierung (''scaling out'')

\subsection{GFS und MapReduce}
\section{Hadoop 1.0 Aufbau}
\subsection{HDFS in depth theoretisch}
\subsection{MapReduce in depth theoretisch}

\section{Hadoop 2.x mit YARN}
\subsection{Neue Möglichkeiten mit YARN}
\subsection{Neuer Aufbau, neue Verpflichtungen}
\subsection{Code einer YARN Application}

\section{Hadoop 3.x anreißen}
\chapter{Das Hadoop Ecosystem}
Im Laufe der Jahre hat sich um den Kern von Hadoop ein reichhaltiges Ökosystem an weiteren Projekten gebildet (das Hadoop Ecosystem). Diese erweitern die Funktionalitäten des Kerns, oder ersetzen gar ganze Komponenten durch alternative Implementierungen. Hadoop ist so modular konzipiert, dass dies problemlos möglich ist. Das Ecosystem lässt sich grob in vier Kategorien aufteilen: Datenspeicherung, Verwaltung~/~Konfiguration, Datentransfer und Datenverarbeitung (vgl. Abb. \ref{fig:ecosys}). Ausgesuchte Komponenten werden in den nächsten Abschnitten kurz vorgestellt.

\begin{figure}[ht]
    \centering
    \includegraphics[width=\textwidth]{Hadoop_ecosystem_overview}
    \caption[Komponenten des Hadoop Ecosystems]{Komponenten des Hadoop Ecosystems\parencite{van_der_weel_hadoop_2015}}
    \label{fig:ecosys}
\end{figure}

\section{Datenhaltung}
\subsection{HDFS}
Das HDFS ist Hadoops mitgeliefertes Dateisystem für die verteilte Speicherung und den parallelen Zugriff auf Daten im Gigabyte- bis Terabyte-Bereich (siehe Kapitel \ref{chap:fund sec:core sub:hdfs}). Dabei stehen hohe Durchsatzraten und Verfügbarkeit im Vordergrund. Echtzeit-Zugriffe auf Daten im Millisekunden-Bereich sind nicht möglich. An Dateien kann außerdem zwar angehängt, aber nicht an beliebigen Stellen etwas eingefügt werden \parencite[vgl.][Kapitel 3]{white_hadoop_2015}. Für solche Anwendungsfälle kann Apache HBase verwendet werden. 
\subsection{HBase}
Apache HBase\footnote{https://hbase.apache.org/} ist eine nichtrelationale, verteilte Datenbank, die auf Hadoop und dem HDFS aufbaut. HBase bietet beliebige Lese- und Schreibzugriffe in Echtzeit auf Tabellen mit Milliarden von Zeilen und Millionen von Spalten. Datensätze werden in \textit{ColumnFamilies} und \textit{Columns} gespeichert, wobei beliebig Spalten hinzugefügt oder weggelassen werden können. HBase speichert solche dünnbesetzten Tabellen sehr effizient, da ein leerer Wert, ein\textit{null}, anders als bei relationalen Datenbanken keinen Festplattenspeicher belegt \parencite[vgl.][Abschnitt 21]{hbase_team_apache_2022}.  

\section{Cluster-Verwaltung / -Konfiguration}
\subsection{YARN}
YARN ist Hadoops Ressourcen-Manager und Job-Tracker (siehe Kapitel \ref{chap:fund sec:core sub:yarn}). Will eine Anwendung auf einem Hadoop Cluster laufen, fordert sie über YARNs API Ressourcen wie Prozessorkerne und Arbeitsspeicher in Form von \textit{Containern} an. YARN stellt diese auf ausgesuchten Nodes im Cluster bereit und abstrahiert den Aspekt der Verteilung. YARN übernimmt außerdem die Abarbeitung von Jobs durch konfigurierbare Warteschlangen. Durch YARN wird es möglich, Applikationen auf einem Hadoop Cluster verteilt laufen zu lassen, die nicht das MapReduce Framework benutzen.\parencite[vgl.][YARN -> Architecture]{noauthor_apache_nodate}
\subsection{Oozie}
Apache Oozie\footnote{https://oozie.apache.org/} dient der Ablaufsteuerung von Hadoop Jobs. Durch Nutzung einer auf XML basierenden \textit{Process Definition Language}\footnote{https://oozie.apache.org/docs/5.2.1/DG\_Overview.html} werden Abhängigkeiten zwischen Hadoop Jobs modelliert, zum Beispiel wenn Job B Daten benötigt, die von Job A generiert werden. Oozie sorgt dafür, dass ein so definierter \textit{Oozie Workflow} sequenziell korrekt abgearbeitet wird. Mit \textit{Oozie Coordinator} kann man außerdem Auslösekriterien für Workflows definieren, wie eine Tageszeit oder die Verfügbarkeit eines neuen Datensatzes. 
\subsection{ZooKeeper}
Apache ZooKeeper\footnote{https://zookeeper.apache.org/} ist ein Service zur zentralen Verwaltung und Synchronisation kleiner Dateien (< 1 MB; Konfigurationsdateien) über einem Cluster. Anwendungen verwenden ZooKeeper für unterschiedliche Aufgaben: HBase überwacht mittels ZooKeeper die Verfügbarkeit seiner RegionServer\footnote{vgl. https://blog.cloudera.com/what-are-hbase-znodes/}. Das HDFS kann im \textit{High Availability Mode} einen sekundären NameNode bereit halten, welcher ohne signifikante Ausfallzeit den NameNode ersetzen kann, sollte dieser ausfallen. Dieser \textit{Automatic Failover} wird von ZooKeeper angestoßen\footnote{https://hadoop.apache.org/docs/stable/hadoop-project-dist/hadoop-hdfs/HDFSHighAvailabilityWithQJM.html}.
\subsection{Ambari}
Apache Ambari\footnote{https://ambari.apache.org/} ermöglicht die Provisionierung, Verwaltung und Überwachung von Apache Hadoop Clustern mittels einer Reihe von übersichtlichen Web Dashboards. Es wurde zum Teil bereits in Kapitel \ref{chap:handson sec:ambari} benutzt. Mit Ambari kann man eine Vielzahl von Hadoop Services im Cluster installieren, konfigurieren und den Zustand des Clusters anhand diverser Metriken überwachen.

\section{Datentransfer}
\subsection{Sqoop}
Apache Sqoop\footnote{https://sqoop.apache.org/} war ein Top-Level Projekt von Apache, bis es im Juni 2021 eingestellt wurde. Es ist noch zum Download verfügbar und kommerzielle Anbieter wie Cloudera bieten weiterhin Support an\footnote{https://community.cloudera.com/t5/Product-Announcements/ANNOUNCE-Apache-Sqoop-Support-on-Cloudera-Data-Platform/m-p/325636/highlight/true\#M339}, aber es wird nicht mehr aktiv weiterentwickelt. Die Hauptaufgabe von Sqoop ist der Massenimport und -Export von Daten zwischen Hadoop und strukturierten Datenquellen wie relationalen Datenbanken. Diese Aufgabe scheint es trotz seiner Einstellung weiterhin als Nischenprodukt zu bedienen, auch wenn Apache Spark ähnliche Funktionalitäten bieten kann.\cite{cloudera_moderator_using_2021} 
\subsection{AVRO}
Apache AVRO\footnote{https://avro.apache.org/} ist ein System zur Datenserialisation. Es ist voll kompatibel mit MapReduce, das heißt die komplexen Datenstrukturen, die man mit AVRO definieren kann, können als Eingabe- und Ausgabetypen für Mapper und Reducer verwendet werden\footnote{https://avro.apache.org/docs/1.11.1/mapreduce-guide/}

\section{Datenverarbeitung}
\subsection{MapReduce}
Hadoop MapReduce ist Hadoops natives Processing Framework, mit dessen Hilfe große, verteilt gespeicherte Datenmengen parallelisiert auf Clustern von tausenden Maschinen verarbeitet werden können. Es wird tiefergehend, mit Anwendungsbeispielen versehen, in Kapitel \ref{chap:fund sec:core sub:mapred} behandelt. MapReduce ist, wie alle Hadoop Core-Komponenten, für den Betrieb auf \textit{Commodity-Hardware} bestimmt. Dadurch werden die Zwischenergebnisse der einzelnen Arbeitsschritte (\textit{Map} und \textit{Reduce}) nicht im Arbeitsspeicher gehalten, sondern auf die Festplatte ausgelagert. Modernere Processing Frameworks wie Apache Spark hingegen nutzen den Umstand, dass RAM immer erschwinglicher wird, und bieten bis zu 40-mal schnellere Verarbeitungszeiten durch \textit{In-Memory-Processing} \cite[vgl.][Kap. 3.19]{freiknecht_big_2018}.
\subsection{Tez}
Apache Tez\footnote{https://tez.apache.org/} ist ein auf YARN aufsetzendes, generalisiertes Framework zur Erstellung von Datenverarbeitungs-Pipelines. Während das MapReduce Framework nur eine Art von Ablauf ermöglicht (Mapper speisen Reducer, wobei zusammen gehörende Daten durch den gleichen Key gekennzeichnet werden; Reducer geben Ergebnisse aus) kann man mit TEZ flexible Abläufe definieren. So kann man Dataflows, die vorher mehrere verkettete MapReduce Jobs benötigt haben, mit einem einzigen TEZ Job erledigen. TEZ unterstützt Batch-Processing und interaktive Datenverarbeitung und kann von Apache Tools wie Hive und Pig komplett als Ersatz für MapReduce verwendet werden \parencite[vgl.][]{noauthor_apache_nodate-2}.
\subsection{Pig}
Apache Pig\footnote{https://pig.apache.org/} dient zum Analysieren großer Datensätze durch eine High-Level Skript-Sprache -- Pig Latin. für ETL und Datenanalyse
generiert MapReduce Jobs
Eine Zeile Pig -> 120 Zeilen MapReduce
Evtl. kleines Beispiel mit Wetterdaten
\subsection{Hive}
Apache Hive ist eine Data Warehousing-Anwendung zum Lesen, Schreiben und Verwalten großer, verteilt gespeicherter Datenmengen. Hive baut dabei auf Hadoop auf und kann Daten verwenden, welche lokal, im HDFS, oder in HBase in gespeichert sind. Daten können in verschiedenen Eingabeformaten vorliegen (CSV, TSV, etc.) und es können eigene Konnektoren geschrieben werden. Nutzer verwenden Hive durch eine Version der weit bekannten Abfragesprache SQL. Abfragen können MapReduce, Apache Spark, oder Apache Tez als Processing Engine verwenden und Hive kann dabei Antwortzeiten von unter einer Sekunde erreichen. Hive skaliert reibungslos mit dem Hadoop Cluster, auf dem es eingerichtet ist.\parencite{noauthor_apache_2020}
\subsection{Spark}
Allround besseres Framework durch In-Memory Processing
Kann mit Hadoop interfacen

\chapter{Hadoop Operations}
\section{Hadoop Setup} 
\subsection{Single Node Setup}
\subsection{Pseudo-Distributed Setup} 
\subsection{Fully-distributed Cluster}
\subsection{Docker Images}
\subsection{VM Distributionen}
\subsection{Hadoop in der Cloud}
HDFS oder Cloud FS 

\section{Praxis}
\subsection{Start eines Single Node Clusters lokal}
\subsection{Erste Übung zum Umgang mit dem HDFS}
\subsection{MapReduce Workflow: Erstellung eines MapReduce Jobs, Kopieren auf den Name Node und Ausführung}
\subsection{YARN Application auf einem Cluster in der Cloud laufen lassen}
\chapter{ETL mit Pig}
Auch wenn es vermehrt von Spark verdrängt wird
\section{Anwendungsfälle}
Welche neuen Dinge ermöglicht dieses Tool
Eine Zeile Pig Latin entspricht vielen Zeilen MapReduce
\section{Architektur}
\section{Pig Latin}
\section{Praxis}
\subsection{Hinzufügen zum Cluster}
\subsection{Anwendung auf dem Cluster}
\chapter{Data Ingestion}
\section{Sqoop}
\section{Flume}
\chapter{Datawarehousing mit Hive}
\section{Anwendungsfälle}
Welche neuen Dinge ermöglicht dieses Tool
\subsection{Unterschiede zu Pig}
\section{Architektur}
\section{Interaktion}
\subsection{HiveQL}
\subsection{CLI}
\subsection{Java API}
\section{Praxis}
\subsection{Hinzufügen zum Cluster}
\subsection{Einrichtung einer Datenbank}
\subsection{Einlesen von Daten im CLI}
\subsection{Einlesen von Daten mit Sqoop}
\subsection{Absetzen einer Query}
\chapter{NoSQL mit HBase}
\section{Anwendungsfälle}
Welche neuen Dinge ermöglicht dieses Tool
\subsection{CAP-Theorem}
\subsection{ACID und BASE}
\section{Architektur}
\section{Interaktion}
\subsection{HBase Shell}
\subsection{Java API}
\section{Praxis}
\subsection{Hinzufügen zum Cluster}
\subsection{Einrichtung einer Datenbank}
\subsection{Einlesen von Daten}
\subsection{Datenmigration aus einem RDBMS}
\subsection{Absetzen einer Query}
\chapter{Streaming mit Kafka}
\section{Anwendungsfälle}
Welche neuen Dinge ermöglicht dieses Tool
Ersetzt durch Spark Streaming
\section{Architektur}
\section{Interaktion}
\subsection{Who knows}
\section{Praxis}
\subsection{Hinzufügen zum Cluster}
\subsection{Maybe, vielleicht kann man ja was zeigen}
\chapter{Hadoop heute}
\section{Aktuelle Anwendungsbeispiele zu Hadoop}
\subsection{AirBnB}
\section{Apache Spark als Gold Standard}
\subsection{Kann eh alles besser}

\backmatter 

\appendix

\printbibliography

\end{document}
% ----------------------------------------------------------------------------
